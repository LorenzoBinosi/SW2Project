\section{Introduction}\label{sec introduction}

\subsection{Purpose}

This is the  Requirement Analysis and Specification Document for the PowerEnJoy car-sharing service.
Its purpose is to completely describe the system, its components, functional and non-functional requirements, constraints, and relationships with the external world, and to provide typical use cases and scenarios for all the actors involved. It is also a strong baseline for project planning and cost estimization, for software evaluation and change control.


This document is written for project managers, developers, testers and systems analysts. Users are not generally interested in detailed software requirements, but they can still find it useful. It may be used in a contractual requirement.

\subsection{Scope}

PowerEnJoy is a digital management system for a car-sharing service that exclusively employs electric vehicles. This car rental system provides an alternative solution to public transport, thus being not only eco-friendly, but also simple and reliable.

The service is available to whomever has a valid driving license. Anybody who wants to register must provide a copy of it, along with his credentials and payment informations.
Once registered, they will be allowed to search a vehicle in a specific area, and reserve one for up to an hour before picking it up. The system will cancel the reservation in case of them failing to arrive to the vehicle in time, will make that specific vehicle available again and apply a fine to the driver.
Upon arrival to the vehicle, the driver will be able to get inside and drive it, at the cost of a per-minute fee of which he will be notified by the system through a screen on the vehicle itself; this until the vehicle will be left parked in a safe area, where the system will close the doors and make the vehicle available again.

A support team is provided by the service in case of incidents, vehicle failures and other cases of need.

\subsection{Actors}
\begin{itemize}
	\item Guest: anyone not recognized by the system who can sign up or log in to it. He/She may look for informations about the service.
	\item Driver: a user subscribed to the system.
	\item Operator: a user that is working for, and subscribed by, the system.
\end{itemize}

\subsection{Goals}
The goals of the PowerEnJoy service are the following:
\begin{itemize}
	\item {[G1]} Allow a driver to rent a vehicle. 
	\begin{itemize}
		\item {[G1.1]} Allow a driver to search and reserve a vehicle. 
		\item {[G1.2]} Allow a driver to use a reserved vehicle. 
		\item {[G1.3]} Allow a driver to park a vehicle in a safe area, suggested by the system.
	\end{itemize} 
	\item {[G2]} Allow a guest to sign up to the system. 
	\item {[G3]} Motivate a driver into car pooling. 
	\item {[G4]} Motivate a driver to a better behavior. 
	\item {[G5]} Provide an efficient and fast assistance in case of problems to vehicles. 
\end{itemize}

\subsection{Definitions, acronyms, and abbreviations}

\begin{description}
	\item[RASD:] Requirements Analysis and Specification Document.
	\item[System:] the whole software system to be developed, comprehensive of all its parts.
	\item[Car pooling:] the sharing of car journeys so that more than one person travels in a car.
	\item[User:] anyone recognized by the system. In this case drivers or operators.
	\item[RDBMS:]
	\item[API:]
	\item[JVM:] 
\end{description}

\subsection{References}
This document refers to the project rules of the Software Engineering 2 project~\cite{se-project-rules} and to the RASD assignment~\cite{se-assignment}.

This document follows the IEEE Standard 830-1998~\cite{ieee-830-1198} for the format of Software Requirements specifications.
\subsection{Overview}
This document is structured in three parts:
\begin{description}
	\item[\autoref{sec introduction}: Introduction.] It provides an overall description of the system scope and purpose, together with some information on this document.
	\item[\autoref{sec overall-desc}: Overall description.] Provides a broad perspective over the principal system features, constraints, and assumptions about the users and the environment.
	\item[\autoref{sec requirements}: Specific requirements.] Goes into detail about functional and nonfunctional requirements. This chapter is arranged by feature.
\end{description}

