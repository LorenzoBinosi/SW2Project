\section{Introduction}
\subsection{Purpose}

This is the  Requirement Analysis and Specification Document for the PowerEnJoy car-sharing service.
Its purpose is to completely describe the system, its components, functional and non-functional requirements, constraints, and relationships with the external world, and to provide typical use cases and scenarios for all the actors involved. It is also a strong baseline for project planning and cost estimization, for software evaluation and change control.


This document is written for project managers, developers, testers and systems analysts. Users are not generally interested in detailed software requirements, but they can still find it useful. It may be used in a contractual requirement.

\subsection{Scope}

PowerEnJoy is a digital management system for a car-sharing service that exclusively employs electric cars.This car rental system provides an alternative solution to public transport, thus being not only eco-friendly, but also simple and reliable.

The service is available to whomever has a valid driving license. Anybody who wants to register must provide a copy of it, along with his credentials and payment informations.

Once registered, they will be allowed to search a car in a specific area, and reserve one for up to an hour before picking it up. The system will cancel the reservation in case of them failing to arrive to the car in time, will make that specific car available again and apply a fine to the user.
Upon arrival to the car, the user will be able to get inside and drive it, at the cost of a per-minute fee of which he will be notified by the system through a screen on the car itself; this until the car will be left parked in a safe area, where the system will close the doors and make the vehicle available again.

\subsection{Actors}
\begin{itemize}
	\item Driver: the only user of the system. He can register and log-in to the system, using it for the car-sharing service.  
\end{itemize}

\subsection{Goals}
The goals of the PowerEnJoy service are the following:
\begin{itemize}
	\item {[G1]}
\end{itemize}