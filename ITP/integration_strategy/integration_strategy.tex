\section{Integration Strategy} \label{sec integration strategy}


\subsection{Entry Criteria}

In order to perform an efficient and reliable integration testing of the software architecture, we must define the prerequisites to be achieved before it can be started.

First of all, both the RASD and the DD must be delivered, and tested according to the following criteria: 

\begin{description}
	\item[completeness:] whether all components are present and described completely.
	\item[consistency:] whether all components do not contradict each other.
	\item[feasibility:] whether the project is feasible considering the given constraints (deadlines, budget and so on).
	\item[testability:] whether or not a system fulfills its requirements.
\end{description}

Afterwards, the integration testing process must be developed following the defined strategy of Section \ref{sec integration test strategy}. 

It's suggested to start the integration process once there is enought code that could be integrated, taking into account the constraints carried by the strategy. For instance, if the development of the mobile application is not started, we could still start the integration process of the web logic. Obviously, it requires the integration of the business logic to be completed and the entire web logic to be written.

\subsection{Elements to be Integrated}

The elements to be integrated are described in the DD and now recalled, highlighting the subsystem that belong to them, in Table \ref{table subsystem 2}.

\begin{table} [H]
    \centering
    \begin{small}
    \begin{tabular}{| p{0.3\textwidth} | p{0.45\textwidth} | p{0.2\textwidth} |}
    \hline
    \textbf{Element} & \textbf{Description} & \textbf{Subsystem} \\
    \hline
    DBMS & The database of the system & Database\\
    \hline
    SafeArea & The EJB of the safe areas & Business\\
    \hline
    Car & The EJB of the cars & Business\\
    \hline
    Rent & The EJB of the rents & Business\\
    \hline
    RentalEvent & The EJB of the rental events & Business\\
    \hline
    CarCredential & The EJB of the cars' credential & Business\\
    \hline
    Driver & The EJB of the drivers & Business\\
    \hline
    CreditCard & The EJB of the credit cards & Business\\
    \hline
    Park & The EJB of the parked cars & Business\\
   \hline
    CarAssistance & The EJB of the cars' issues reports & Business\\
    \hline
    DriverManager & The session EBJ for the user's operations & Business\\
    \hline
    DriverRentalManager & The session EBJ for the rental operations & Business\\
    \hline
    SearchCarManager & The session EBJ for the searching operations & Business\\
    \hline
    LogBookManager & The session EBJ for the logbook operations & Business\\
    \hline
    CarHeartbeat & The session EBJ for the car heartbeat management & Business\\
    \hline
    RentalFee & The singleton EBJ for the rental fee  & Business\\
    \hline
    CarRentalManager & The session EBJ for the car rental information & Business\\
    \hline
    Servlets & All the servlets that operate using the session EJBs & Business\\
    \hline
    ServletContainer & The servlet container that invoke the right servlet once called & Business\\
    \hline
    AndroidUIManager & The Android UI manager & AndroidMobile\\
    \hline
    iOSUIManager & The iOs UI manager & iOSMobile\\
    \hline
    AndroidGPSManager & The Android GPS manager & AndroidMobile\\
    \hline
    iOSGPSManager & The iOS GPS manager & iOSMobile\\
    \hline
    AndroidController & The Android controller & AndroidMobile\\
    \hline
    iOSController & The iOS controller & iOSMobile\\
    \hline
    WebContoller & The web controller that communicates with the application server & Web\\
    \hline
    JSP & The JSP of the web application & Web\\
    \hline
    WebContainer & The web container that invoke the right JSP once called & Web\\
    \hline
    \end{tabular}
    \end{small}
    \label{table subsystem 1}
\end{table}


\begin{table} [H]
    \centering
    \begin{small}
    \begin{tabular}{| p{0.3\textwidth} | p{0.45\textwidth} | p{0.2\textwidth} |}
    \hline
    CarApplication & The car application that notifies the system about the car/renta status & Car\\
    \hline
    CarGUI & The car application that supports several utilities (Radio, GPS Navigator, ...) & Car\\
    \hline
    \end{tabular}
    \end{small}
    \caption{Elements of the PowerEnJoy system to be integrated.}
    \label{table subsystem 2}
\end{table}
  
\subsection{Integration Testing Strategy} \label{sec integration test strategy}

The strategy adopted for the integration testing is the \textbf{bottom-up} approach, for the software integration, and the \textbf{critical-module-first} approach, for the subsystem integration. The choice of these approaches it's due to the fact that the integration process of a subsystem is automatically completed once the software of the same subsystem is integrated.

Moreover, the performance of the system could be evaluated at each integration step using \textbf{JMeter}. In fact if the performance drop down it's probably due to the last integrated element/subsystem.

\subsection{Sequence of Component/Function Integration}

\subsubsection{Software Integration Sequence} \label{software integration}

The software integration sequence is summarized in Table \ref{elements sequence 2}.

Choosing the bottom-up approach the elements are tested from the most indipendent to the less one. They require the implementation of some drivers which are developed using \textbf{Mockito}, aimed to tests the iteraction between elements of the same subsystem, and \textbf{Arquillian}, aimed to tests the iteraction beetwen elements of different subsystem. Both frameworks require \textbf{JUnit} for some essential testing functionalities (for instance the \textit{assertEquals} function).

\begin{table}[H]
    \centering
    \begin{small}
    \begin{tabular}{| l | l | p{0.35\textwidth} | p{0.3\textwidth} |}
    \hline
    \textbf{N.} & \textbf{Subsystem} & \textbf{Component} & \textbf{Integrates with} \\
    \hline
    I1 & Database, Business & (JEB) Car & DBMS\\
    \hline
    I2 & Database, Business & (JEB) Rent & DBMS\\
    \hline
    I3 & Database, Business & (JEB) Driver & DBMS\\
    \hline
    I4 & Database, Business & (JEB) Park & DBMS\\
   \hline
    I5 & Database, Business & (JEB) SafeArea & DBMS\\
    \hline
    I6 & Database, Business & (JEB) RentalEvent & DBMS\\
    \hline
    I7 & Database, Business & (JEB) CreditCard & DBMS\\
    \hline
     I8 & Database, Business & (JEB) CarAssistance & DBMS\\
    \hline
     I9 & Database, Business & (JEB) CarCredential & DBMS\\
    \hline
     I10 & Business & (SB) RentalFee & Settings.xml \\
    \hline
     I11 & Business & (SB) DriverManager & Driver \newline CreditCard\\
    \hline
     I12 & Business & (SB) DriverRentalManager & Driver \newline Car \newline Rent \newline CarCredential \newline Park\\
    \hline
     I13 & Business & (SB) SearchCarManager & Driver \newline Car\\
    \hline
     I14 & Business & (SB) LogBookManager & Driver \newline Rent  \\
    \hline
     I15 & Business & (SB) CarHeatbeat & CarCredential \newline Rent \newline Car  \newline CreditCard  \newline Park  \newline CarAssistance  \newline RentalEvent  \newline RentalFee  \\
    \hline
     I16 & Business & (SB) CarRentalManager & CarCredential \newline Driver \newline SafeArea  \newline Park \newline RentalFee  \newline RentalEvent \\
    \hline
    I17  & Business & Servlets & DriverManager \newline DriverRentalManager \newline SearchCarManager \newline LogBookManager \newline CarHeatbeat \newline CarRentalManager\\
    \hline
    \end{tabular}
    \end{small}
    \label{elements sequence 1}
\end{table}

\begin{table}[H]
    \centering
    \begin{small}
    \begin{tabular}{| l | l | p{0.35\textwidth} | p{0.3\textwidth} |}
    \hline
    I18 & Business & ServletContainer & Servlets \\
    \hline
    I19  & Web & WebController & (Business) ServletContainer\\
    \hline
    I20 & Web & JSP & WebController\\
    \hline
    I21 & Web & WebContainer & JSP\\
    \hline
    I22  & Mobile & iOSController & (Business) ServletContainer\\
    \hline
    I23  & Mobile & AndroidController & (Business) ServletContainer\\
    \hline
    I24  & Mobile & iOSGPSManager & iOSController\\
    \hline
    I25  & Mobile & AndroidGPSManager & AndroidController\\
    \hline
    I26  & Mobile & iOSUIManager & iOSController\\
    \hline
    I27  & Mobile & AndroidUIManager & AndroidController\\
    \hline   
    I28  & Car & CarApplication & (Business) ServletContainer\\
    \hline   
    I29  & Car & CarGUI & (Business) ServletContainer\\
    \hline 
    \end{tabular}
    \end{small}
    \caption{Integration of the PowerEnJoy system's elements.}
    \label{elements sequence 2}
\end{table}

\subsubsection{Subsystem Integration Sequence} \label{subsystem integration}

As already said, the strategy adopted for the subsystem integration is the critical-module-first approach. For the chosen software architecture, this approach allow us to a better and easier integration. The database requires simple drivers in order to be integrate, instead, the business logic doesn't need them at all. In fact it's possible to test the correct integration of the business subsytem just making API calls in an automated fashion.

The subsystem integration sequence is summerized in Table \ref{subsystems sequence}.

\begin{table}[H]
    \centering
    \begin{tabular}{| l | l | l |}
        \hline
        \textbf{N.} & \textbf{Subsystem} & \textbf{Integrates with} \\
        \hline
        SI1 & Business & Database\\
        SI2 & Web & Business\\
        SI3 & Mobile & Business \\
        SI4 & Car & Business \\
        \hline
    \end{tabular}
    \caption{Integration of the PowerEnJoy's subsystems.}
    \label{subsystems sequence}
\end{table}