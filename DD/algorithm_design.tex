\section{Algorithm Design} \label{sec algorithm design}

The proposed algorithms are the most interesting, and less intuitive, algorithms of the business logic. We won't be using any reference to the Java EE components in order to look at these algorithms in the most abstract way possible. 
The convention used for them contains:
 \begin{itemize}
	\item $variables$.
	\item {\color{blue}not\_implemented\_methods()}: intuitive methods that often interact with other components like the database or the car.
	\item {\color{green}implemented\_methods()}: methods described later.
	\item {\color{red} COSTANTS}.
\end{itemize}

\subsection{Car Heartbeat}

The car hearbeat management is a core function of the businnes logic. It checks the cars' status and execute actions based on their status.
 
The associated request provides the following infomation to the application server:
\begin{description}
\item [JSON Object:] it contains:
	\begin{itemize}
		\item $carInformation$: the information about the car.
		\item $rentInformation$: the information about the current rent.
	\end{itemize}
\item [Path Parameters:] they are:
	\begin{itemize}
		\item $carID$: the car ID used to get authenticate by server.
		\item $carPassword$: the car's password used to get authenticate by server.
	\end{itemize}
\end{description}
The algorithm used to manage the car hearbeat is the following.

\bigskip

\begin{algorithm}[H]
\small
	\eIf{\upshape{({\color{blue}verifyCredentials} ($carID$, $carPassword$))}}
	{
		{\color{green} updateCarStatus} ($carID$, $carInformation$); \\
		\If{\upshape{({\color{blue}rentedCar} ($carID$))}}
		{
			{\color{green} updateRentalStatus} ($carID$, $rentInformation$); \\
		}
		HTTPResponseCode({\color{red}200});
	}
	{
		HTTPResponseCode({\color{red}403});
	}
\end{algorithm}

\subsubsection{Car status updating}

The parameters of this function are:
\begin{itemize}
	\item $carInformation$: the information about the car.
	\item $carID$: the car ID used to get authenticate by server.
\end{itemize}
The $carInformation$ object contains:
\begin{itemize}
	\item $batteryLevel$: the battery level of the car.
	\item $position$: an object that contains the geographical postion of the car.
	\item $brokenParts[]$: the list of the broken parts of the car.
\end{itemize}
The $position$ object contains:
\begin{itemize}
	\item $latitude$: the latitude value.
	\item $longitude$: the longitude value..
\end{itemize}
The algorithm is the following.

\bigskip

\begin{algorithm}[H]
\small
	{\color{blue} updateCarValues}($carID$, $batteryLevel$, $position$); \\ 
	\If{\upshape{($brokenParts$ != null)}}
	{
		\If{\upshape{({\color{blue}getOpenedAssistanceReport} ($carID$, {\color{red}105}) == null)}}
		{
			{\color{blue} createAssistanceReport} ($carID$, {\color{red}105}, brokenParts[]); \\
		} 
	}
	\eIf{\upshape{($batteryLevel$ == 0)}}
	{
		\If{\upshape{({\color{blue}getOpenedAssistanceReport} ($carID$, {\color{red}103}) == null)}}
		{
			{\color{blue}createAssistanceReport} ($carID$, {\color{red}103}); \\
		} 
	}
	{
		\If{\upshape{($batteryLevel$ \textless= 25  \&\& !{\color{blue}rentedCar} ($carID$))}}
		{
			\If{\upshape{({\color{blue}getOpenedAssistanceReport} ($carID$, {\color{red}104}) == null)}}
			{
				{\color{blue} setCarOutOfService} ($carID$); \\
				{\color{blue} createAssistanceReport} ($carID$, {\color{red}104}); \\
			} 
		}
	}
\end{algorithm}

\subsubsection{Rental status updating}

The parameters of this function are:
\begin{itemize}
	\item $rentInformation$: the information about the current rent.
	\item $carID$: the car ID used to get authenticate by server.
\end{itemize}
The $carInformation$ object contains:
\begin{itemize}
	\item $events[]$: the list of rental events.
	\item $isEmpty$: is true if there is no one in the car.
	\item $isLockable$: is true if the car is turned off and closed correctly.
\end{itemize}
The algorithm is the following.

\bigskip

\begin{algorithm}[H]
\small
	$rentID$ = {\color{blue} getRentID} ($carID$ ); \\
	\If{\upshape{($events$ != null)}}
	{
		{\color{blue}updateEvents} ($rentID$, $events$); \\
	}
	\If{\upshape{($isLockable$ \&\& $isEmpty$)}}
	{
		\If{\upshape{({\color{blue}inASafeArea} ($carID$))}}
		{
			{\color{blue}closeCar} ($carID$); \\
			{\color{blue}closeRent} ($rentID$); \\
			{\color{green}calculateCost} ($rentID$); \\
			{\color{blue}checkCarStatus} ($carID$);  \/\/Similar to updateCarStatus without 1st line \\
			{\color{blue}executePayment} ($rentID$); \\
		}
	} 
	Other error cases...
\end{algorithm}

\subsubsection{Rental cost calculation}

The parameters of this function are:
\begin{itemize}
	\item $rentID$: the rental ID.
\end{itemize}
The algorithm is the following.
\bigskip

\begin{algorithm}[H]
\small
	$endTime$ = {\color{blue} getRentalEndDate} ($rentID$); \\
	$beginTime$ = {\color{blue}getFirstRentalEventByType} ($rentID$, {\color{red}TURNED\_ON}); \\
	\If{\upshape{($beginTime$ == null)}}
	{
		$beginTime$  = {\color{blue}getRentalBeginDate} ($rentID$);\\
	}
	$rentalTime$ = ($endTime$  -  $beginTime$); \\
	$totalCost$ = ($rentalTime$ : 60) * {\color{blue}getFee}(); \\
	$passengerTimes[]$ =  {\color{blue} orderArraysByTime} ({\color{blue}getRentalEventsByType} ($rentID$, {\color{red}PASSENGERS\_DETECTED}), {\color{blue}getRentalEventsByType} ($rentID$, {\color{red}ALL\_PASSENGERS\_LEFT})); 
	
	\If{\upshape{($passengerTimes$ != null)}}
	{
		$passengerTotalTime$ = 0; \\
		\For{\upshape{$i$ = 0 \textbf{to} $passengerTimes$.length()}}{$passengerTotalTime$ = $passengerTotalTime$  + ($passengerTimes[i + 1]$ - $passengerTimes[i]$); \\
			$i$ = $i$ + 2; \\
		}
		\If{\upshape{($passengerTotalTime$ \textgreater= ($rentalTime$ * 0.5))}}
		{
			$totalCost$ = $totalCost$ * 0.8; \\
		} 
	}
	{\color{blue}setCost} ($rentID$, $totalCost$);
\end{algorithm}