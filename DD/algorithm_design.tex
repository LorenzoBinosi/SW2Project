\section{Algorithm Design} \label{sec algorithm design}

\subsection{Car Heartbeat}

The car hearbeat management is a core function of the businnes logic. It checks the cars' status and execute actions based on their status.
 
The associated request provides the following infomation to the application server:
\begin{description}
\item [JSON Object:] it contains:
	\begin{itemize}
		\item $carInformation$: the information about the car.
		\item $rentInformation$: the information about the current rent.
	\end{itemize}
\item [Path Parameters:] they are:
	\begin{itemize}
		\item $carID$: the car ID used to get authenticate by server.
		\item $carPassword$: the car's password used to get authenticate by server.
	\end{itemize}
\end{description}
The algorithm used to manage the car hearbeat is the following.

\bigskip

\begin{algorithm}[H]
\small
	\eIf{\upshape{({\color{blue}verifyCredentials} ($carID$, $carPassword$))}}
	{
		{\color{green} updateCarStatus} ($carID$, $carInformation$); \\
		\If{\upshape{({\color{blue}rentedCar} ($carID$))}}
		{
			{\color{green} updateRentalStatus} ($carID$, $rentInformation$); \\
		}
		HTTPResponseCode({\color{red}200});
	}
	{
		HTTPResponseCode({\color{red}403});
	}
\end{algorithm}

\subsubsection{Car status updating}

The parameters of this function are:
\begin{itemize}
	\item $carInformation$: the information about the car.
	\item $carID$: the car ID used to get authenticate by server.
\end{itemize}
The $carInformation$ object contains:
\begin{itemize}
	\item $batteryLevel$: the battery level of the car.
	\item $position$: an object that contains the geographical postion of the car.
	\item $brokenParts[]$: the list of the broken parts of the car.
\end{itemize}
The $position$ object contains:
\begin{itemize}
	\item $latitude$: the latitude value.
	\item $longitude$: the longitude value..
\end{itemize}
The algorithm is the following.

\bigskip

\begin{algorithm}[H]
\small
	{\color{blue} updateCarValues}($carID$, $batteryLevel$, $position$); \\ 
	\If{\upshape{($brokenParts$ != null)}}
	{
		\If{\upshape{({\color{blue}getOpenedAssistanceReport} ($carID$, {\color{red}105}) == null)}}
		{
			{\color{blue} createAssistanceReport} ($carID$, {\color{red}105}, brokenParts[]); \\
		} 
	}
	\eIf{\upshape{($batteryLevel$ == 0)}}
	{
		\If{\upshape{({\color{blue}getOpenedAssistanceReport} ($carID$, {\color{red}103}) == null)}}
		{
			{\color{blue}createAssistanceReport} ($carID$, {\color{red}103}); \\
		} 
	}
	{
		\If{\upshape{($batteryLevel$ \textless= 25  \&\& !{\color{blue}rentedCar} ($carID$))}}
		{
			\If{\upshape{({\color{blue}getOpenedAssistanceReport} ($carID$, {\color{red}104}) == null)}}
			{
				{\color{blue} setCarOutOfService} ($carID$); \\
				{\color{blue} createAssistanceReport} ($carID$, {\color{red}104}); \\
			} 
		}
	}
\end{algorithm}

\subsubsection{Rental status updating}

The parameters of this function are:
\begin{itemize}
	\item $rentInformation$: the information about the current rent.
	\item $carID$: the car ID used to get authenticate by server.
\end{itemize}
The $carInformation$ object contains:
\begin{itemize}
	\item $events[]$: the list of rental events.
	\item $isEmpty$: is true if there is no one in the car.
	\item $isLockable$: is true if the car is turned off and closed correctly.
\end{itemize}
The algorithm is the following.

\bigskip

\begin{algorithm}[H]
\small
	$rentID$ = {\color{blue} getRentID} ($carID$ ); \\
	\If{\upshape{($events$ != null)}}
	{
		{\color{blue}updateEvents} ($rentID$, $events$); \\
	}
	\uIf{\upshape{($isLockable$ \&\& $isEmpty$)}}
	{
		\uIf{\upshape{(getRentalEvent($rentID$, TURNED_OFF) != null \&\& inASafeArea($carID$))}}
		{
			closeCar($carID$);
			closeRent($rentID$);
			calculateCost($rentID$);
			checkCarStatus($carID$);  //Similar to updateCarStatus without 1st line
			executePayment($rentID$);
		}
	}
	Other error cases...
\end{algorithm}